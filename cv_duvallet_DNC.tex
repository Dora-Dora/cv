%%%%%%%%%%%%%%%%%%%%%%%%%%%%%%%%%%%%%%%%%%%%%%%%%%%%%%%%%%%%%%%%%%%%%%%%
%%%%%%%%%%%%%%%%%%%%%% Simple LaTeX CV Template %%%%%%%%%%%%%%%%%%%%%%%%
%%%%%%%%%%%%%%%%%%%%%%%%%%%%%%%%%%%%%%%%%%%%%%%%%%%%%%%%%%%%%%%%%%%%%%%%

%%%%%%%%%%%%%%%%%%%%%%%%%%%%%%%%%%%%%%%%%%%%%%%%%%%%%%%%%%%%%%%%%%%%%%%%
%% NOTE: If you find that it says                                     %%
%%                                                                    %%
%%                           1 of ??                                  %%
%%                                                                    %%
%% at the bottom of your first page, this means that the AUX file     %%
%% was not available when you ran LaTeX on this source. Simply RERUN  %%
%% LaTeX to get the ``??'' replaced with the number of the last page  %%
%% of the document. The AUX file will be generated on the first run   %%
%% of LaTeX and used on the second run to fill in all of the          %%
%% references.                                                        %%
%%%%%%%%%%%%%%%%%%%%%%%%%%%%%%%%%%%%%%%%%%%%%%%%%%%%%%%%%%%%%%%%%%%%%%%%

%%%%%%%%%%%%%%%%%%%%%%%%%%%% Document Setup %%%%%%%%%%%%%%%%%%%%%%%%%%%%

% Don't like 10pt? Try 11pt or 12pt
\documentclass[10pt]{article}

% The automated optical recognition software used to digitize resume
% information works best with fonts that do not have serifs. This
% command uses a sans serif font throughout. Uncomment both lines (or at
% least the second) to restore a Roman font (i.e., a font with serifs).
%\usepackage{times}
%\renewcommand{\familydefault}{\sfdefault}

% This is a helpful package that puts math inside length specifications
\usepackage{calc}
\usepackage{comment}

% Simpler bibsection for CV sections
% (thanks to natbib for inspiration)
\makeatletter
\newlength{\bibhang}
\setlength{\bibhang}{1em} %1em}
\newlength{\bibsep}
 {\@listi \global\bibsep\itemsep \global\advance\bibsep by\parsep}
\newenvironment{bibsection}%
        {\begin{enumerate}{}{%
%        {\begin{list}{}{%
       \setlength{\leftmargin}{\bibhang}%
       \setlength{\itemindent}{-\leftmargin}%
       \setlength{\itemsep}{\bibsep}%
       \setlength{\parsep}{\z@}%
        \setlength{\partopsep}{0pt}%
        \setlength{\topsep}{0pt}}}
        {\end{enumerate}\vspace{-.6\baselineskip}}
%        {\end{list}\vspace{-.6\baselineskip}}
\makeatother

% Layout: Puts the section titles on left side of page
\reversemarginpar

%
%         PAPER SIZE, PAGE NUMBER, AND DOCUMENT LAYOUT NOTES:
%
% The next \usepackage line changes the layout for CV style section
% headings as marginal notes. It also sets up the paper size as either
% letter or A4. By default, letter was used. If A4 paper is desired,
% comment out the letterpaper lines and uncomment the a4paper lines.
%
% As you can see, the margin widths and section title widths can be
% easily adjusted.
%
% ALSO: Notice that the includefoot option can be commented OUT in order
% to put the PAGE NUMBER *IN* the bottom margin. This will make the
% effective text area larger.
%
% IF YOU WISH TO REMOVE THE ``of LASTPAGE'' next to each page number,
% see the note about the +LP and -LP lines below. Comment out the +LP
% and uncomment the -LP.
%
% IF YOU WISH TO REMOVE PAGE NUMBERS, be sure that the includefoot line
% is uncommented and ALSO uncomment the \pagestyle{empty} a few lines
% below.
%

%% Use these lines for letter-sized paper
\usepackage[paper=letterpaper,
            %includefoot, % Uncomment to put page number above margin
            marginparwidth=1in,     % Length of section titles
            marginparsep=.03in,       % Space between titles and text
            margin=1in,               % 1 inch margins
            includemp]{geometry}

%% Use these lines for A4-sized paper
%\usepackage[paper=a4paper,
%            %includefoot, % Uncomment to put page number above margin
%            marginparwidth=30.5mm,    % Length of section titles
%            marginparsep=1.5mm,       % Space between titles and text
%            margin=25mm,              % 25mm margins
%            includemp]{geometry}

%% More layout: Get rid of indenting throughout entire document
\setlength{\parindent}{0in}

\usepackage[shortlabels]{enumitem}

%% Reference the last page in the page number
%
% NOTE: comment the +LP line and uncomment the -LP line to have page
%       numbers without the ``of ##'' last page reference)
%
% NOTE: uncomment the \pagestyle{empty} line to get rid of all page
%       numbers (make sure includefoot is commented out above)
%
\usepackage{fancyhdr,lastpage}
\pagestyle{fancy}
%\pagestyle{empty}      % Uncomment this to get rid of page numbers
\fancyhf{}\renewcommand{\headrulewidth}{0pt}
\fancyfootoffset{\marginparsep+\marginparwidth}
\newlength{\footpageshift}
\setlength{\footpageshift}
          {0.5\textwidth+0.5\marginparsep+0.5\marginparwidth-2in}
\lfoot{\hspace{\footpageshift}%
       \parbox{4in}{\, \hfill %
                    \arabic{page} of \protect\pageref*{LastPage} % +LP
%                    \arabic{page}                               % -LP
                    \hfill \,}}

% Finally, give us PDF bookmarks
\usepackage{color,hyperref}
\definecolor{darkblue}{rgb}{0.0,0.0,0.3}
\hypersetup{colorlinks,breaklinks,
            linkcolor=darkblue,urlcolor=darkblue,
            anchorcolor=darkblue,citecolor=darkblue}

%%%%%%%%%%%%%%%%%%%%%%%% End Document Setup %%%%%%%%%%%%%%%%%%%%%%%%%%%%


%%%%%%%%%%%%%%%%%%%%%%%%%%% Helper Commands %%%%%%%%%%%%%%%%%%%%%%%%%%%%

% The title (name) with a horizontal rule under it
% (optional argument typesets an object right-justified across from name
%  as well)
%
% Usage: \makeheading{name}
%        OR
%        \makeheading[right_object]{name}
%
% Place at top of document. It should be the first thing.
% If ``right_object'' is provided in the square-braced optional
% argument, it will be right justified on the same line as ``name'' at
% the top of the CV. For example:
%
%       \makeheading[\emph{Curriculum vitae}]{Your Name}
%
% will put an emphasized ``Curriculum vitae'' at the top of the document
% as a title. Likewise, a picture could be included:
%
%   \makeheading[\includegraphics[height=1.5in]{my_picutre}]{Your Name}
%
% the picture will be flush right across from the name.
\newcommand{\makeheading}[2][]%
        {\hspace*{-\marginparsep minus \marginparwidth}%
         \begin{minipage}[t]{\textwidth+\marginparwidth+\marginparsep}%
             {\large \bfseries #2 \hfill #1}\\[-0.15\baselineskip]%
                 \rule{\columnwidth}{1pt}%
         \end{minipage}}

% The section headings
%
% Usage: \section{section name}
\renewcommand{\section}[1]{\pagebreak[3]%
    \hyphenpenalty=10000%
    \vspace{1.3\baselineskip}%
    \phantomsection\addcontentsline{toc}{section}{#1}%
    \noindent\llap{\scshape\smash{\parbox[t]{\marginparwidth}{\raggedright #1}}}%
    \vspace{-\baselineskip}\par}

% An itemize-style list with lots of space between items
\newenvironment{outerlist}[1][\enskip\textbullet]%
        {\begin{itemize}[#1,leftmargin=*]}{\end{itemize}%
         \vspace{-.6\baselineskip}}

% An environment IDENTICAL to outerlist that has better pre-list spacing
% when used as the first thing in a \section
\newenvironment{lonelist}[1][\enskip\textbullet]%
        {\begin{list}{#1}{%
        \setlength{\partopsep}{0pt}%
        \setlength{\topsep}{0pt}}}
        {\end{list}\vspace{-.6\baselineskip}}

% An itemize-style list with little space between items
\newenvironment{innerlist}[1][\enskip\textbullet]%
        {\begin{itemize}[#1,leftmargin=*,parsep=0pt,itemsep=0pt,topsep=0pt,partopsep=0pt]}
        {\end{itemize}}

% An environment IDENTICAL to innerlist that has better pre-list spacing
% when used as the first thing in a \section
\newenvironment{loneinnerlist}[1][\enskip\textbullet]%
        {\begin{itemize}[#1,leftmargin=*,parsep=0pt,itemsep=0pt,topsep=0pt,partopsep=0pt]}
        {\end{itemize}\vspace{-.6\baselineskip}}

% To add some paragraph space between lines.
% This also tells LaTeX to preferably break a page on one of these gaps
% if there is a needed pagebreak nearby.
\newcommand{\blankline}{\quad\pagebreak[3]}
\newcommand{\halfblankline}{\quad\vspace{-0.5\baselineskip}\pagebreak[3]}

% Uses hyperref to link DOI
\newcommand\doilink[1]{\href{http://dx.doi.org/#1}{#1}}
\newcommand\doi[1]{doi:\doilink{#1}}

% For \url{SOME_URL}, links SOME_URL to the url SOME_URL
\providecommand*\url[1]{\href{#1}{#1}}
% Same as above, but pretty-prints SOME_URL in teletype fixed-width font
\renewcommand*\url[1]{\href{#1}{\texttt{#1}}}

% For \email{ADDRESS}, links ADDRESS to the url mailto:ADDRESS
\providecommand*\email[1]{\href{mailto:#1}{#1}}
% Same as above, but pretty-prints ADDRESS in teletype fixed-width font
%\renewcommand*\email[1]{\href{mailto:#1}{\texttt{#1}}}

%\providecommand\BibTeX{{\rm B\kern-.05em{\sc i\kern-.025em b}\kern-.08em
%    T\kern-.1667em\lower.7ex\hbox{E}\kern-.125emX}}
%\providecommand\BibTeX{{\rm B\kern-.05em{\sc i\kern-.025em b}\kern-.08em
%    \TeX}}
\providecommand\BibTeX{{B\kern-.05em{\sc i\kern-.025em b}\kern-.08em
    \TeX}}
\providecommand\Matlab{\textsc{Matlab}}

%%%%%%%%%%%%%%%%%%%%%%%% End Helper Commands %%%%%%%%%%%%%%%%%%%%%%%%%%%

%%%%%%%%%%%%%%%%%%%%%%%%% Begin CV Document %%%%%%%%%%%%%%%%%%%%%%%%%%%%

\begin{document}
\makeheading{Claire Duvallet}

\section{}

% NOTE: Mind where the & separators and \\ breaks are in the following
%       table.
%
% ALSO: \rcollength is the width of the right column of the table
%       (adjust it to your liking; default is 1.85in).
%
\newlength{\rcollength}\setlength{\rcollength}{1.4in}%

%\begin{tabular}[t]{@{}p{\textwidth-\rcollength}p{\rcollength}}
\begin{tabular}[t]{@{}p{\textwidth-\rcollength}p{\rcollength}}

\email{cduvallet@gmail.com} & 9 Seattle St \\
\href{https://cduvallet.github.io/}{cduvallet.github.io} & Allston, MA 02134 \\

%9 Seattle St   & 512-636-9548 & cduvallet.github.io \\
%Allston, MA  02134     & \email{duvallet@mit.edu} & \\
\end{tabular}

\section{}

I am a PhD in computational biology interested in using data science to impact society.

%Insert text here if you want to
%\begin{innerlist}
%\item More information and auxiliary documents can be found at\\\url{http://www.tedpavlic.com/facjobsearch/}
%\end{innerlist}

%\section{Research Interests}

%systems biology, evolution, ecology, population genetics, -omics technologies, information theory, nonlinear dynamics, stochastic processes, probability theory, data mining, statistical learning, bayesian inference

\section{Skills}

\textbf{Technical}: Python (pandas, seaborn, matplotlib, scikit-learn, etc); git and GitHub; AWS (S3, EC2, Glacier, IAM); bash; LaTeX; R (proficient).

\textbf{Non-technical}: conflict management; student advocacy; project management; oral and written communication; diversity, equity, and inclusion; leadership, empowerment, and continuity.

\section{Education}

\textbf{Massachusetts Institute of Technology}, \textit{Cambridge, MA} \hfill {2014 -- 2019} \\
Ph.D., Biological Engineering, January 2019
%\begin{outerlist}
%\item[] Ph.D., Biological Engineering \\
%\textit{GPA}: 5.0/5.0
%\end{outerlist}
\vspace{.1in}

\textbf{Columbia University}, \textit{New York, NY} \hfill {2009 -- 2013} \\
B.S., Biomedical Engineering
%\begin{outerlist}
%\item[] B.S., Biomedical Engineering, \textit{cell and tissue engineering track} \\
%\textit{GPA}: 4.1/4.0, \textit{Summa cum laude}
%\end{outerlist}

\section{Research}

\textbf{Massachusetts Institute of Technology} \hfill {2015 -- 2019}\\
Supervisor: Eric J. Alm, Ph.D.  \\
Ph.D., \textit{Department of Biological Engineering} (2015 -- Jan. 2019) \\
Postdoc (Feb. 2019 -- Apr. 2019)
  \vspace{0.08in}

I studied the relationship between the microbiome and health and disease, mining large clinical and biological datasets to extract scientific insight. \\

%\begin{innerlist}
%\item{Performed a meta-analysis of 28 case-control gut microbiome studies across 10 disease states}
%\item{Characterized the lung, gastric, and oropharyngeal microbiomes of pediatric patients with impaired swallow function}
%\item{Proposed a framework for donor selection in fecal microbiota transplant clinical trials}
%\item{Mining untargeted metabolomics data from residential sewage to identify human-derived biomarkers}
%\item{Developing methods to process and analyze untargeted metabolomics of blood for precision diagnostics and outcome prediction}
%\end{innerlist}

%\vspace{.08in}

\textbf{Columbia University} \hfill {2011 -- 2013} \\
Supervisor: Samuel L. Sia, Ph.D. \\
\textit{Molecular and Microscale Bioengineering Laboratory}
	\vspace{0.08in}

As an undergraduate research assistant, I worked on developing a point-of-care microfluidic device to diagnose multi-drug resistant tuberculosis. 

%\begin{innerlist}
%\item{Designed and developed the DNA amplification and detection modules of point-of-care diagnostic}
%\item{Optimized primers, detection probes, reagents, and reaction conditions for multiplex, fast, and isothermal PCR reactions}
%\end{innerlist}

%\vspace{.15in}

%\textbf{Ecole Polytechnique} \hfill {Summer 2012}\\
%Supervisor: Cedric Norais, Ph.D. \\
%\textit{Laboratoire de Biochimie}
%	\vspace{0.08in}
%
%As an international undergraduate research intern, I studied the acquired-immunity CRISPR system in E. coli. \\

%\begin{innerlist}
%\item{Constructed plasmids to study components of the CRISPR system in \textit{E. coli} and expressed and purified CasBCD*E protein complex}
%\end{innerlist}

\section{Publications}
\vspace{-.1275in}
\begin{bibsection}
	\item Chengzhen Dai, \textbf{Claire Duvallet}, An Ni Zhang, Mariana Matus, Newsha Ghaeli, Shinkyu Park, Noriko Endo, Siavash Isazadeh, Tong Zhang, Kazi Jamil, Carlo Ratti, and Eric Alm. \\
	``Multi-site sampling and risk prioritization reveals the public health relevance of antibiotic resistance genes found in sewage environments.'' \\
	\textit{submitted} (\href{https://www.biorxiv.org/content/10.1101/562496v1}{bioRxiv preprint}).
	
	\item \textbf{Claire Duvallet}, Caroline Zellmer, Pratik Panchal, Shrish Budree, Madji Osman, and Eric Alm. \\ 
		``Framework for rational donor selection in fecal microbiota transplant clinical trials.'' \\ 
		\textit{submitted}.
		
	\item Mariana Matus, \textbf{Claire Duvallet}, Newsha Ghaeli, Melissa Kido Soule, Krista Longnecker, Ilana Brito, Carlo Ratti, Elizabeth B. Kujawinski, and Eric Alm. \\ 
		``Untargeted detection of human health and activity markers in residential wastewater through microbiome sequencing and metabolomics.'' \\ 
		\textit{in preparation}.
		
	\item \textbf{Claire Duvallet}, Kara Larson, Scott Snapper, Sonia Iosim, Ann Lee, Katherine Freer, Kara May, Eric Alm, and Rachel Rosen. \\ 
		``Aerodigestive sampling reveals altered microbial exchange between lung, oropharyngeal, and gastric microbiomes in children with impaired swallow function.'' (2019) \\ 
		\textit{PLoS ONE}, doi: \href{https://doi.org/10.1371/journal.pone.0216453}{10.1371/journal.pone.0216453}.
	
	\item Keegan Korthauer$^{*}$, Patrick Kimes$^{*}$, \textbf{Claire Duvallet}$^{\dagger}$,  Alejandro Reyes$^{\dagger}$,  Ayshwarya Subramanian$^{\dagger}$, Mingxiang Teng, Chinmay Shukla, Eric Alm, and Stephanie Hicks. \\
		``A practical guide to methods controlling false discoveries in computational biology.''  \\
		\textit{Genome Biology}, \textit{accepted} (\href{https://www.biorxiv.org/content/early/2018/10/31/458786}{bioRxiv preprint}).
		
	\item Sean Gibbons, \textbf{Claire Duvallet}, and Eric Alm. (2018) \\ 
		``Correcting for batch effects in case-control microbiome studies.'' \\ 
		\emph{PLoS Computational Biology}. doi: \href{https://doi.org/10.1371/journal.pcbi.1006102}{10.1371/journal.pone.0176335}.
		
	\item \textbf{Claire Duvallet}. (2018) \\ 
		``Meta-analysis generates and prioritizes hypotheses for translational microbiome research.'' \\ 
		\emph{Microbial Biotechnology}. doi: \href{https://doi.org/10.1111/1751-7915.13047}{10.1111/1751-7915.13047}.
		
	\item \textbf{Claire Duvallet}, Sean Gibbons, Thomas Gurry, Rafael  Irizarry, and Eric Alm. (2017) \\ 
		``Meta-analysis of gut microbiome studies identifies disease-specific and shared responses.'' \\ 
		\emph{Nature Communications}. doi: \href{https://doi.org/10.1038/s41467-017-01973-8}{10.1038/s41467-017-01973-8}.
		
		\begin{itemize}
			\item Received PSB Award for Rigorous Secondary Data Analysis 
		\end{itemize}		
				
	\item Scott Olesen, \textbf{Claire Duvallet}, and Eric Alm. (2017) \\ 
		``dbOTU3: A new implementation of distribution-based OTU calling.'' \\ 
		\emph{PloS ONE}. doi: \href{https://doi.org/10.1371/journal.pone.0176335}{10.1371/journal.pone.0176335}.

	\item \textit{[Non-peer reviewed blog post]} \textbf{Claire Duvallet}. (2019) \\ 
		``Scientific discovery from a clinical study: surprises from the lung and stomach microbiomes.'' 
		\emph{Nature Microbiology Community Forum}. (\href{https://go.nature.com/30rx4VZ}{go.nature.com/30rx4VZ}).
		
	\item \textit{[Non-peer reviewed blog post]} \textbf{Claire Duvallet}. (2017) \\ 
		``Beyond dysbiosis: disease-specific and shared microbiome responses to disease.'' 
		\emph{Nature Microbiology Community Forum}. (\href{http://go.nature.com/2As9meL}{go.nature.com/2As9meL}).
		
	\item \textit{[Dataset]} \textbf{Claire Duvallet}, Sean Gibbons, Thomas Gurry, Rafael Irizarry, and Eric Alm. (2017) \\ 
		``MicrobiomeHD: the human gut microbiome in health and disease.'' \\ 
		\emph{Zenodo}. doi: \href{https://zenodo.org/record/1146764}{10.5281/zenodo.1146764}
\end{bibsection}

\section{ }
%\vspace{-.1in}
%\textbf{Oral presentations}
%\begin{bibsection}
%\item ``Framework for rational donor selection in fecal microbiota transplant clinical trials.'' International Conference on Microbiome Engineering (ICME 2018). Boston, MA. Nov. 2018. \textit{Invited}.
%\item ``Predictive power of the microbiome.'' Science on Tap! Boston College Department of Biology seminar series. Boston, MA. Aug. 2018. \textit{Invited}.
%\item ``Distribution-based methods to increase power and reduce redundancy in microbiome data.'' Teaching and Developing QIIME 2 Workshop. San Diego, CA. May 2018. \textit{Selected}.
%\item ``Meta-analysis to identify consistent disease-associated microbiome shifts.'' MIT-Harvard Microbiome Symposium. Cambridge, MA. March 2018. \textit{Selected}.
%%\item ``Meta-analysis to identify consistent disease-associated microbiome shifts.'' MIT Department of Biological Engineering Retreat. Cambridge, MA. October 2017. \textit{Invited}.
%\end{bibsection}
%\vspace{0.15in}

%\textbf{Poster presentations}
%\begin{bibsection}
%\item ``Meta-analysis of gut microbiome studies identifies disease-specific and shared responses.'' \textit{Women in Data Science Cambridge}, March 2018 and \textit{Pacific Symposium on Biocomputing}, January 2018
%\item ``Empirical signatures of compositional stability in the gut microbiome.'' \textit{Statistical and Algorithmic Challenges in Microbiome Data Analysis Workshop}, MIT Center for Informatics and Therapeutics and The Simons Center for Data Analysis, February 2016.
%\end{bibsection} 

\section{Software}

\textbf{Percentile normalization} \\
Correcting batch effects in case-control microbiome studies. \\
\begin{innerlist}
	\item[] Python implementation: \href{https://github.com/seangibbons/percentile_normalization}{github.com/seangibbons/percentile\_normalization} \textit{(contributor)}
	\item[] QIIME 2 plugin: \href{https://github.com/cduvallet/q2-perc-norm}{github.com/cduvallet/q2-perc-norm} \textit{(developer)}
\end{innerlist}
\vspace{.15in}

\textbf{Distribution-based OTU calling} \\
New implementation of Preheim \textit{et al}.'s distribution-based OTU clustering algorithm. \\
\begin{innerlist}
	\item [] Python implementation: \href{https://github.com/swo/dbotu3}{github.com/swo/dbotu3} \textit{(contributor)}
    \item [] QIIME 2 plugin: \href{https://github.com/cduvallet/q2-dbotu}{github.com/cduvallet/q2-dbotu} \textit{(developer)}
\end{innerlist}
\vspace{.15in}

\textbf{Amplicon sequencing pipeline} \\
End-to-end pipeline to process 16S data. \\
\begin{innerlist}
	\item [] Python: \href{https://github.com/thomasgurry/amplicon_sequencing_pipeline}{github.com/thomasgurry/amplicon\_sequencing\_pipeline} \textit{(co-developer)}
\end{innerlist}
\vspace{.15in}

% Add a little space to nudge next ``Conference Publications'' marginpar
% down to make room for tall ``Submitted Journal Publications''
% marginpar. If there are enough submitted journal publications, this
% space will not be needed (and should be removed).
%\vspace{0.1in}

\section{Fellowships \& Awards}

\textbf{Fellowships}
\begin{outerlist}
	\item[] Siebel Scholars Foundation \hfill {2018} \\
		\textit{Siebel Scholar Class of 2019}
	\item[] National Defense Science and Engineering Graduate Fellowship \hfill {2015 -- 2018} \\
		\textit{NDSEG Recipient}
	\item[] National Science Foundation Graduate Research Fellowship \hfill {2015} \\ 
		\textit{Honorable Mention} 
	\item[] Henry Luce Foundation \hfill {2013 -- 2014} \\ 
		\textit{Luce Scholar} 
\end{outerlist}
\vspace{.15in}

\textbf{Awards}
\begin{outerlist}
	\item[] PSB Award for Rigorous Secondary Data Analysis \hfill{2019} \\
		\textit{Junior Research Parasite}
	\item[] MIT Graduate Women of Excellence \hfill {2017}
	\item[] Salutatorian \hfill {2013} \\
		\textit{Columbia University Fu Foundation School of Engineering and Applied Science}
%	\item[] Richard Skalak Award in Biomedical Engineering \hfill {2013} \\
%		\textit{Columbia University Department of Biomedical Engineering}
%	\item[] Robert E. and Claire S. Reiss Prize in Biomedical Engineering \hfill {2013} \\
%		\textit{Columbia University Department of Biomedical Engineering} 
	\item[] King's Crown Bronze Leadership Award \hfill {2012} \\
		\textit{Columbia University}
%	\item[] Tau Beta Pi, The Engineering Honor Society \hfill {2012}
%	\item[] Valedictorian\hfill {2009} \\
%		\textit{James Bowie High School}, ranked first out of 636 students
\end{outerlist}

%\vspace{.15in}


\section{Teaching Experience}

\textbf{Teaching Assistant} \hfill {Fall 2015} \\
20.106 Systems Microbiology, \textit{Massachusetts Institute of Technology} 
  \vspace{0.08in}

I was a TA for seven advanced undergraduate students in a new course on the human microbiome, emerging disease, phylogenetics, and host-microbe interactions. \\

%\begin{innerlist}
%\item{Developed problem sets and guided lecture content for a module on processing and analyzing 16S data}
%\item{Facilitated and participated in paper discussions on various topics in zoonotic disease, viral communities, host immune responses, and the human microbiome}
%\item{Mentored students on project re-processing and re-analyzing published microbiome datasets with machine learning tools}
%\end{innerlist}

%\vspace{.15in}

\textbf{Lecturer} \hfill {2013 -- 2014} \\
Biomedical Equipment Technology Department, \textit{University of Puthisastra} \\
Engineering World Health, \textit{Phnom Penh, Cambodia} 
  \vspace{0.08in}

As a Luce Scholar, I was one of the first lecturers for Engineering World Health's new Associate Bachelor’s program in Biomedical Equipment Technology at the University of Puthisastra, a private university in Phnom Penh. 

%\begin{innerlist}
%\item{Developed curricula and syllabi for anatomy and physiology, troubleshooting skills, and math modules}
%\item{Prepared and delivered lectures, exams, assignments, in-class activities, and demos for two classes of 12-18 Cambodian students and technicians (in English, with a translator)}
%\item{Managed Cambodian student-teaching staff during main supervisor's absence and supported foreign and local teaching staff}
%\end{innerlist}

%\vspace{.15in}

%\textbf{Teaching Assistant} \hfill {2012 -- 2013} \\
%The Art of Engineering, \textit{Columbia University} 
%  \vspace{0.08in}
%
%My senior year at Columbia, I TAed the biomedical engineering section of the introductory engineering course for freshmen. \\

%\begin{innerlist}
%\item{Assisted students in designing and building a vital-signs monitoring device, guiding them through the engineering design process}
%\item{Taught concepts in MATLAB and circuitry required for projects}
%\end{innerlist}

%\vspace{.15in}

\section{Leadership \\ \& Service}

\textbf{Academic and professional}
\begin{outerlist}
	\item[] \href{https://twitter.com/MITubiomeclub}{MIT Microbiome Club} \hfill {2015 -- 2018} 
	\begin{innerlist}
		\item[] \textit{Co-Founder, President, Executive board member}
		\item[] \textit{Co-Founder and organizing committee}, MIT-Harvard Microbiome Symposium
	\end{innerlist}
			
	\item[] \href{http://biotech.mit.edu/}{MIT Biotech Group} \hfill {2017 -- 2018} \\ 
		\textit{Beyond the Bench Initiative board member} 
\end{outerlist}
\vspace{.15in}

\textbf{Departmental and MIT}
\begin{outerlist}
	\item[] Graduate Student Advisory Group for Engineering (GradSAGE) \hfill {2017 -- 2018} \\
	Advisory group to the Dean of the School of Engineering
%	\textit{Advisor/Advisee Relations Sub-Committee}

	\item[] Biological Engineering Department Visiting Committee \hfill {2018} \\
		\textit{Graduate student representative}

	\item[] MIT Graduate Student Council \hfill {2017 -- 2019} \\
		Diversity and Inclusion Subcommittee  \\
		\textit{Vice Chair, Department and Classroom Inclusion co-coordinator} 

	\item[] \href{http://berefs.com/}{BE Resources for Easing Friction and Stress (BE REFS)} \hfill {2016 -- 2019} \\
	\textit{Confidential conflict management coach and graduate student advocate}
	
	\begin{innerlist} 
		\item[] \textit{Lead author}, \href{http://berefs.com/wp-content/uploads/2018/03/Grad-Support-Flowchart-MIT-Digital.pdf}{Grad Support Resources Flowchart}
	\end{innerlist}

	\item[] BE Graduate Student Board \hfill {2015 -- 2018} \\ 
		\textit{Diversity Chair}
		\begin{innerlist} 
			\item[] \textit{Co-Founder}, \href{http://be.mit.edu/academic-programs/prospective-graduate/beaap}{BE Application Assistance Program}
			\item[] \textit{Lead author}, \href{http://be.mit.edu/about/department-values-statement}{BE Departmental Values Statement}
			\item[] \textit{Co-lead}, 2016 BE Diversity Survey
		\end{innerlist}
\end{outerlist}
\vspace{.15in}

\textbf{Outreach \& Mentorship}
\begin{outerlist}

	\item[] MIT Microbiome superUROP \hfill {2016 -- 2017} \\
		\textit{Mentor, supervised one undergraduate researcher}

	\item[] Science Club for Girls, Young Leaders in STEM \hfill {2016 and 2017} \\ \textit{Volunteer, developed and taught three-day course on} \\ \textit{microbiology and the human microbiome} 

	\item[] MIT SPLASH \hfill 2015 \\
		\textit{Volunteer instructor, ``Microbiome 101: What's in your poop?''}

	\item[] E3: Empowering, Encouraging, and Eliminating Barriers for \\ Women in STEM \hfill 2015 \\
		\textit{Mentor (2015), guest presenter (2016)}

\item[] ESL Program for MIT Service Employees \hfill {2015 -- 2018} \\
	\textit{Math GED tutor}

\end{outerlist}
\vspace{.15in}

%\textbf{Reviewer}
%\begin{outerlist}
%	\item[] MIT Summer Research Program (MSRP) Reviewer \hfill {2018} 
%	\item[] MIT Committed to Caring Selection Committee \hfill {2017} 
%	\item[] MIT IDEAS Global Challenge Reviewer \hfill {2015 -- present} 
%	\end{outerlist}
%\vspace{.15in}

%\section{Extra-curriculars}
%
%\begin{outerlist}
%\item[] {MIT Interfaith Dialogue Program} \hfill {2017 -- 2018} \\
%\textit{Addir fellow}
%\item[] {sMITe, MIT Women's Ultimate Frisbee Team} \hfill {2014 -- present} \\
%\textit{B-team captain (2014 -- 2015)}
%\end{outerlist}

\end{document}

%%%%%%%%%%%%%%%%%%%%%%%%%% End CV Document %%%%%%%%%%%%%%%%%%%%%%%%%%%%%

%----------------------------------------------------------------------%
% The following is copyright and licensing information for
% redistribution of this LaTeX source code; it also includes a liability
% statement. If this source code is not being redistributed to others,
% it may be omitted. It has no effect on the function of the above code.
%----------------------------------------------------------------------%
% Copyright (c) 2007, 2008, 2009, 2010, 2011 by Theodore P. Pavlic
%
% Unless otherwise expressly stated, this work is licensed under the
% Creative Commons Attribution-Noncommercial 3.0 United States License. To
% view a copy of this license, visit
% http://creativecommons.org/licenses/by-nc/3.0/us/ or send a letter to
% Creative Commons, 171 Second Street, Suite 300, San Francisco,
% California, 94105, USA.
%
% THE SOFTWARE IS PROVIDED "AS IS", WITHOUT WARRANTY OF ANY KIND, EXPRESS
% OR IMPLIED, INCLUDING BUT NOT LIMITED TO THE WARRANTIES OF
% MERCHANTABILITY, FITNESS FOR A PARTICULAR PURPOSE AND NONINFRINGEMENT.
% IN NO EVENT SHALL THE AUTHORS OR COPYRIGHT HOLDERS BE LIABLE FOR ANY
% CLAIM, DAMAGES OR OTHER LIABILITY, WHETHER IN AN ACTION OF CONTRACT,
% TORT OR OTHERWISE, ARISING FROM, OUT OF OR IN CONNECTION WITH THE
% SOFTWARE OR THE USE OR OTHER DEALINGS IN THE SOFTWARE.
%----------------------------------------------------------------------%
